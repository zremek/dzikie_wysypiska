% Options for packages loaded elsewhere
\PassOptionsToPackage{unicode}{hyperref}
\PassOptionsToPackage{hyphens}{url}
%
\documentclass[
]{article}
\usepackage{amsmath,amssymb}
\usepackage{lmodern}
\usepackage{iftex}
\ifPDFTeX
  \usepackage[T1]{fontenc}
  \usepackage[utf8]{inputenc}
  \usepackage{textcomp} % provide euro and other symbols
\else % if luatex or xetex
  \usepackage{unicode-math}
  \defaultfontfeatures{Scale=MatchLowercase}
  \defaultfontfeatures[\rmfamily]{Ligatures=TeX,Scale=1}
\fi
% Use upquote if available, for straight quotes in verbatim environments
\IfFileExists{upquote.sty}{\usepackage{upquote}}{}
\IfFileExists{microtype.sty}{% use microtype if available
  \usepackage[]{microtype}
  \UseMicrotypeSet[protrusion]{basicmath} % disable protrusion for tt fonts
}{}
\makeatletter
\@ifundefined{KOMAClassName}{% if non-KOMA class
  \IfFileExists{parskip.sty}{%
    \usepackage{parskip}
  }{% else
    \setlength{\parindent}{0pt}
    \setlength{\parskip}{6pt plus 2pt minus 1pt}}
}{% if KOMA class
  \KOMAoptions{parskip=half}}
\makeatother
\usepackage{xcolor}
\IfFileExists{xurl.sty}{\usepackage{xurl}}{} % add URL line breaks if available
\IfFileExists{bookmark.sty}{\usepackage{bookmark}}{\usepackage{hyperref}}
\hypersetup{
  pdftitle={Dzikie wysypiska w Łodzi. Wyniki badań},
  pdfauthor={Remek},
  hidelinks,
  pdfcreator={LaTeX via pandoc}}
\urlstyle{same} % disable monospaced font for URLs
\usepackage[margin=1in]{geometry}
\usepackage{graphicx}
\makeatletter
\def\maxwidth{\ifdim\Gin@nat@width>\linewidth\linewidth\else\Gin@nat@width\fi}
\def\maxheight{\ifdim\Gin@nat@height>\textheight\textheight\else\Gin@nat@height\fi}
\makeatother
% Scale images if necessary, so that they will not overflow the page
% margins by default, and it is still possible to overwrite the defaults
% using explicit options in \includegraphics[width, height, ...]{}
\setkeys{Gin}{width=\maxwidth,height=\maxheight,keepaspectratio}
% Set default figure placement to htbp
\makeatletter
\def\fps@figure{htbp}
\makeatother
\setlength{\emergencystretch}{3em} % prevent overfull lines
\providecommand{\tightlist}{%
  \setlength{\itemsep}{0pt}\setlength{\parskip}{0pt}}
\setcounter{secnumdepth}{-\maxdimen} % remove section numbering
\usepackage{booktabs}
\usepackage{longtable}
\usepackage{array}
\usepackage{multirow}
\usepackage{wrapfig}
\usepackage{float}
\usepackage{colortbl}
\usepackage{pdflscape}
\usepackage{tabu}
\usepackage{threeparttable}
\usepackage{threeparttablex}
\usepackage[normalem]{ulem}
\usepackage{makecell}
\usepackage{xcolor}
\ifLuaTeX
  \usepackage{selnolig}  % disable illegal ligatures
\fi

\title{Dzikie wysypiska w Łodzi. Wyniki badań}
\author{Remek}
\date{2022-07-25}

\begin{document}
\maketitle

{
\setcounter{tocdepth}{2}
\tableofcontents
}
\hypertarget{wyniki-w-zakresie} wpisów (50 z 183).
\end{itemize}

Stan badań na 2022-06-30

\hypertarget{liczba-wpisuxf3w-dziennie}{%
\subsection{Liczba wpisów dziennie}\label{liczba-wpisuxf3w-dziennie}}

\includegraphics{data_exploration_files/figure-latex/wpis-dzien-1.pdf}

\hypertarget{liczba-wpisuxf3w-miesiux119cznie}{%
\subsection{Liczba wpisów
miesięcznie}\label{liczba-wpisuxf3w-miesiux119cznie}}

\includegraphics{data_exploration_files/figure-latex/wpis-miesiac-1.pdf}

\hypertarget{liczba-ochotniczek-ochotnikuxf3w-w-czasie}{%
\subsection{Liczba ochotniczek / ochotników w
czasie}\label{liczba-ochotniczek-ochotnikuxf3w-w-czasie}}

\includegraphics{data_exploration_files/figure-latex/liczba-ochotn-1.pdf}

Łącznie wpisy na mapę dzikich wysypisk przesłały \textbf{54 osoby}.

\hypertarget{aktywnoux15bux107-ochotnikuxf3w-ochotniczek}{%
\subsection{Aktywność ochotników /
ochotniczek}\label{aktywnoux15bux107-ochotnikuxf3w-ochotniczek}}

Wszystkie nazwy ochotników ujednoliciłem do małych liter. Usunąłem
spacje na początku i na końcu nazw. To pozwoliło liczyć nazwy wpisane
np. ``sebastian33444'' i ``Sebastian33444'' jako to samo.

Nazwy moje (Remka) są dwie: Rmo i Uffo7. Wcześniej korzystałem z Uffo7,
potem na potrzeby nagrywania instrukcji wymyśliłem inną nazwę aby
pokazać w formularzu ścieżkę przesyłania wpisu po raz pierwszy.

\hypertarget{iloux15bux107-wpisuxf3w-na-konkretnego-ochotnika}{%
\subsubsection{Ilość wpisów na konkretnego
ochotnika}\label{iloux15bux107-wpisuxf3w-na-konkretnego-ochotnika}}

\begin{table}
\centering
\begin{tabular}[t]{l|r}
\hline
volunteer\_id & n\\
\hline
katarzyna wasilewska & 16\\
\hline
pkrupnik & 15\\
\hline
rmo & 15\\
\hline
hpruszyn & 14\\
\hline
alex1313 & 13\\
\hline
email\_dzikiewysypiska & 11\\
\hline
kraterek & 9\\
\hline
ktk & 8\\
\hline
anna\_krzynowek & 5\\
\hline
darkerone & 5\\
\hline
langel & 5\\
\hline
uffo7 & 4\\
\hline
kolija & 3\\
\hline
quks & 3\\
\hline
radkel & 3\\
\hline
taranah & 3\\
\hline
twardy & 3\\
\hline
zwiedzamłódź & 3\\
\hline
alex & 2\\
\hline
efem & 2\\
\hline
gargl & 2\\
\hline
jul\_woj & 2\\
\hline
maciej & 2\\
\hline
ms1996 & 2\\
\hline
radosław antosik & 2\\
\hline
sebastian33444 & 2\\
\hline
tooomasz & 2\\
\hline
ainka & 1\\
\hline
alexxx19 & 1\\
\hline
atom & 1\\
\hline
bogusia & 1\\
\hline
dim79 & 1\\
\hline
ezamar & 1\\
\hline
j13 & 1\\
\hline
jerzy1956 & 1\\
\hline
jmp & 1\\
\hline
juha & 1\\
\hline
juliaróża & 1\\
\hline
juliaz & 1\\
\hline
kacperka1 & 1\\
\hline
kasia chojnacka & 1\\
\hline
katarzyna & 1\\
\hline
magpas & 1\\
\hline
małgo & 1\\
\hline
małgorzata & 1\\
\hline
nittka969 & 1\\
\hline
noyes & 1\\
\hline
nukazet & 1\\
\hline
ps & 1\\
\hline
romek1076 & 1\\
\hline
sejti & 1\\
\hline
sierra & 1\\
\hline
tomektramwaj & 1\\
\hline
ula urszula & 1\\
\hline
\end{tabular}
\end{table}

\hypertarget{ile-osuxf3b-zrobiux142o-jeden-wpis-ile-wiux119cej}{%
\subsubsection{Ile osób zrobiło jeden wpis? Ile
więcej?}\label{ile-osuxf3b-zrobiux142o-jeden-wpis-ile-wiux119cej}}

Mamy dużo osób, które zrobiły mało wpisów, a mało takich które zrobiły
dużo.

To jest też typowe np. w użytkowaniu mediów społecznościowych (mało osób
ma wysoką aktywność a dużą ma małą).

\begin{table}
\centering
\begin{tabular}[t]{r|r}
\hline
liczba\_wpis & liczba\_ochotnik\\
\hline
1 & 27\\
\hline
2 & 9\\
\hline
3 & 6\\
\hline
4 & 1\\
\hline
5 & 3\\
\hline
8 & 1\\
\hline
9 & 1\\
\hline
11 & 1\\
\hline
13 & 1\\
\hline
14 & 1\\
\hline
15 & 2\\
\hline
16 & 1\\
\hline
\end{tabular}
\end{table}

\hypertarget{iloux15bux107-wpisuxf3w-a-dux142ugoux15bux107-zaangaux17cowania-w-projekt}{%
\subsubsection{Ilość wpisów a długość zaangażowania w
projekt}\label{iloux15bux107-wpisuxf3w-a-dux142ugoux15bux107-zaangaux17cowania-w-projekt}}

Zobaczymy jeszcze, czy ilość wpisów jest powiązana z tym, jak długo
ochotnik jest z nami. Biorę datę zatwierdzaną przez ochotnika.

Liczę różnicę dat jako ostatnia\_data - pierwsza\_data + 1. Czyli 1
dzień oznacza, że ochotnik przesłał wpisy tylko jednego dnia.

\begin{table}
\centering
\begin{tabular}[t]{l|r|r}
\hline
volunteer\_id & liczba\_wpis & pierwszy\_ostatni\_wpis\_ile\_dni\\
\hline
katarzyna wasilewska & 16 & 96\\
\hline
pkrupnik & 15 & 51\\
\hline
rmo & 15 & 57\\
\hline
hpruszyn & 14 & 11\\
\hline
alex1313 & 13 & 16\\
\hline
email\_dzikiewysypiska & 11 & 49\\
\hline
kraterek & 9 & 28\\
\hline
ktk & 8 & 101\\
\hline
anna\_krzynowek & 5 & 7\\
\hline
darkerone & 5 & 44\\
\hline
langel & 5 & 2\\
\hline
uffo7 & 4 & 28\\
\hline
kolija & 3 & 1\\
\hline
quks & 3 & 1\\
\hline
radkel & 3 & 1\\
\hline
taranah & 3 & 24\\
\hline
twardy & 3 & 2\\
\hline
zwiedzamłódź & 3 & 1\\
\hline
alex & 2 & 26\\
\hline
efem & 2 & 4\\
\hline
gargl & 2 & 19\\
\hline
jul\_woj & 2 & 1\\
\hline
maciej & 2 & 19\\
\hline
ms1996 & 2 & 1\\
\hline
radosław antosik & 2 & 1\\
\hline
sebastian33444 & 2 & 37\\
\hline
tooomasz & 2 & 1\\
\hline
ainka & 1 & 1\\
\hline
alexxx19 & 1 & 1\\
\hline
atom & 1 & 1\\
\hline
bogusia & 1 & 1\\
\hline
dim79 & 1 & 1\\
\hline
ezamar & 1 & 1\\
\hline
j13 & 1 & 1\\
\hline
jerzy1956 & 1 & 1\\
\hline
jmp & 1 & 1\\
\hline
juha & 1 & 1\\
\hline
juliaróża & 1 & 1\\
\hline
juliaz & 1 & 1\\
\hline
kacperka1 & 1 & 1\\
\hline
kasia chojnacka & 1 & 1\\
\hline
katarzyna & 1 & 1\\
\hline
magpas & 1 & 1\\
\hline
małgo & 1 & 1\\
\hline
małgorzata & 1 & 1\\
\hline
nittka969 & 1 & 1\\
\hline
noyes & 1 & 1\\
\hline
nukazet & 1 & 1\\
\hline
ps & 1 & 1\\
\hline
romek1076 & 1 & 1\\
\hline
sejti & 1 & 1\\
\hline
sierra & 1 & 1\\
\hline
tomektramwaj & 1 & 1\\
\hline
ula urszula & 1 & 1\\
\hline
\end{tabular}
\end{table}

Na wykresie trochę widać zależność monotoniczną -- jak rośnie jedno to
drugie też. Jest przy tym dużo pojedynczych wpisów.

\includegraphics{data_exploration_files/figure-latex/liczba-wpis-a-czas-wykres-1.pdf}

Skala logarytmiczna żeby ``rozgciągnąć'' niskie wartości, których jest
większość:

\includegraphics{data_exploration_files/figure-latex/liczba-wpis-a-czas-log-1.pdf}

Zobaczmy to z trendem i nazwami ochotników:

\includegraphics{data_exploration_files/figure-latex/liczba-wpis-a-czas-log-text-1.pdf}

Zobaczmy to samo ale tylko dla tych, co mają więcej, niż 1 wpis:

\includegraphics{data_exploration_files/figure-latex/liczba-wpis-a-czas-log-text-1-1.pdf}

Ci co są dłużej niż 10 dni mają więcej wpisów?

\includegraphics{data_exploration_files/figure-latex/liczba-czas-10-1.pdf}

To są nierówne grupy. Tych, co są dłużej mamy znacznie mniej.

\begin{verbatim}
## pierwszy_ostatni_wpis_ile_dni > 10
## FALSE  TRUE 
##    39    15
\end{verbatim}

\begin{verbatim}
## 
##  Spearman's rank correlation rho
## 
## data:  volunt_difftime$pierwszy_ostatni_wpis_ile_dni and volunt_difftime$liczba_wpis
## S = 5140.3, p-value = 2.453e-13
## alternative hypothesis: true rho is not equal to 0
## sample estimates:
##       rho 
## 0.8040663
\end{verbatim}

Korelacja wychodzi 0.8040663 czyli tak, im dłużej ochotnik jest z nami
tym ma więcej wpisów.

Może najsensowniej byłoby to policzyć tylko dla osób, mających więcej,
niż 1 wpis bo takie z 1 wpisem mają zawsze 1 dzień?

\begin{verbatim}
## 
##  Spearman's rank correlation rho
## 
## data:  volunt_difftime_1$pierwszy_ostatni_wpis_ile_dni and volunt_difftime_1$liczba_wpis
## S = 1415.2, p-value = 0.001998
## alternative hypothesis: true rho is not equal to 0
## sample estimates:
##   rho 
## 0.568
\end{verbatim}

Tutaj \textbf{współczynnik jest niższy}, niż kiedy braliśmy wszystkich.

\hypertarget{rozmieszczenie-mapa}{%
\subsection{Rozmieszczenie -- mapa}\label{rozmieszczenie-mapa}}

\includegraphics{data_exploration_files/figure-latex/rozmieszczenie-mapa-1.pdf}

Jeden punkt na mapie oznacza jeden wpis. Łącznie 172 punkty, 94\% z 183
wpisów.

\hypertarget{rozmieszczenie-liczba-w-dzielnicach}{%
\subsection{Rozmieszczenie -- liczba w
dzielnicach}\label{rozmieszczenie-liczba-w-dzielnicach}}

\includegraphics{data_exploration_files/figure-latex/rozmieszczenie-dzielnica-1.pdf}

\hypertarget{rozmieszczenie-liczba-w-dzielnicach-w-odniesieniu-do-powierzchni}{%
\subsection{Rozmieszczenie -- liczba w dzielnicach w odniesieniu do
powierzchni}\label{rozmieszczenie-liczba-w-dzielnicach-w-odniesieniu-do-powierzchni}}

\includegraphics{data_exploration_files/figure-latex/rozmieszczenie-dzielnica-mkw-1.pdf}

\hypertarget{cechy-dzikich-wysypisk-iloux15bux107-informacji}{%
\subsection{Cechy dzikich wysypisk -- ilość
informacji}\label{cechy-dzikich-wysypisk-iloux15bux107-informacji}}

\begin{table}
\centering
\begin{tabular}[t]{l|r|r}
\hline
Podano informację o wysypisku & n & \%\\
\hline
Nie & 133 & 73\\
\hline
Tak & 50 & 27\\
\hline
SUMA & 183 & 100\\
\hline
\end{tabular}
\end{table}

O ile nie wskazano inaczej, \textbf{cechy wysypisk dotyczą 50 wpisów}.

\hypertarget{iloux15bux107-informacji-vs.-uczestnik-todo}{%
\subsubsection{Ilość informacji vs.~uczestnik
TODO}\label{iloux15bux107-informacji-vs.-uczestnik-todo}}

Czy podanie informacji o wysypisku zależy od tego, kto robi wpis? Czy
zależy od tego, który jest to kolejny wpis?

\includegraphics{data_exploration_files/figure-latex/info-a-uczestnik-1.pdf}

Są osoby zawsze wpisujące informacje, są takie, które tego nie robią.
Jak to wyjaśnić? Nie wiem, czy mamy coś sensownego w danych na ten
temat.

Może sama liczba wpisów?

\includegraphics{data_exploration_files/figure-latex/liczba-wpis-a-info-1.pdf}

Nie widzę zależności.

\includegraphics{data_exploration_files/figure-latex/liczba-wpis-info-boxpl-1.pdf}

Też nic nie widać.

Jeszcze zobaczymy frakcję ``tak'' w zależności od pogrupowanej liczby
wpisów

\begin{verbatim}
## 
## FALSE  TRUE 
##    36    18
\end{verbatim}

\includegraphics{data_exploration_files/figure-latex/frc-tak-liczba-wpis-1.pdf}

\begin{verbatim}
##  volunteer_id            Nie              Tak         liczba_wpis_sum 
##  Length:54          Min.   : 0.000   Min.   :0.0000   Min.   : 1.000  
##  Class :character   1st Qu.: 0.000   1st Qu.:0.0000   1st Qu.: 1.000  
##  Mode  :character   Median : 1.000   Median :0.5000   Median : 1.500  
##                     Mean   : 2.463   Mean   :0.9259   Mean   : 3.389  
##                     3rd Qu.: 2.000   3rd Qu.:1.0000   3rd Qu.: 3.000  
##                     Max.   :15.000   Max.   :6.0000   Max.   :16.000  
##     frc_tak       
##  Min.   :0.00000  
##  1st Qu.:0.00000  
##  Median :0.03125  
##  Mean   :0.36263  
##  3rd Qu.:1.00000  
##  Max.   :1.00000
\end{verbatim}

Nie widzę zależności i jest to mało intuicyjne.

TODO - potraktować wpisy jako sekwencję po dacie, pierwszy, drugi itp.
spróbować na tym to sprawdzić

\hypertarget{cechy-dzikich-wysypisk-powierzchnia}{%
\subsection{Cechy dzikich wysypisk --
powierzchnia}\label{cechy-dzikich-wysypisk-powierzchnia}}

\includegraphics{data_exploration_files/figure-latex/cechy-powierzchnia-1.pdf}

\hypertarget{powierzchnia-na-mapach}{%
\subsubsection{Powierzchnia na mapach}\label{powierzchnia-na-mapach}}

\begin{verbatim}
## 'data.frame':    183 obs. of  35 variables:
##  $ ec5_uuid                         : chr  "1e640a49-7e2f-4efa-a188-a3e935b01d08" "23898937-08cd-4e19-9ab6-892b83efff03" "fd37c2a2-bf35-431d-9e6e-a68af049369a" "90bdcd8b-d072-4fc1-b017-ff29cdb5f26c" ...
##  $ created_at                       : chr  "2022-06-16T08:56:59.433Z" "2022-06-16T08:52:33.087Z" "2022-06-12T14:00:19.052Z" "2022-06-10T10:34:43.386Z" ...
##  $ uploaded_at                      : chr  "2022-06-16T08:57:06.000Z" "2022-06-16T08:52:38.000Z" "2022-06-12T14:00:25.000Z" "2022-06-10T10:34:50.000Z" ...
##  $ title                            : chr  "16/06/2022 10:54:39 Ula Urszula" "16/06/2022 10:50:56" "12/06/2022 16:00:03" "10/06/2022 12:33:07" ...
##  $ 1_Prosimy_o_zdjcie_z             : chr  "https://five.epicollect.net/api/media/dzikie-wysypiska?type=photo&format=entry_original&name=1e640a49-7e2f-4efa"| __truncated__ "https://five.epicollect.net/api/media/dzikie-wysypiska?type=photo&format=entry_original&name=23898937-08cd-4e19"| __truncated__ "https://five.epicollect.net/api/media/dzikie-wysypiska?type=photo&format=entry_original&name=fd37c2a2-bf35-431d"| __truncated__ "https://five.epicollect.net/api/media/dzikie-wysypiska?type=photo&format=entry_original&name=90bdcd8b-d072-4fc1"| __truncated__ ...
##  $ 2_Jeli_chcesz_dodaj_             : chr  "https://five.epicollect.net/api/media/dzikie-wysypiska?type=photo&format=entry_original&name=1e640a49-7e2f-4efa"| __truncated__ "" "https://five.epicollect.net/api/media/dzikie-wysypiska?type=photo&format=entry_original&name=fd37c2a2-bf35-431d"| __truncated__ "" ...
##  $ 3_Jeli_chcesz_dodaj_             : chr  "https://five.epicollect.net/api/media/dzikie-wysypiska?type=photo&format=entry_original&name=1e640a49-7e2f-4efa"| __truncated__ "" "" "" ...
##  $ 4_Jeli_chcesz_dodaj_             : chr  "" "" "" "" ...
##  $ 6_Jeeli_masz_uwagi_d             : chr  "Park źródliska II" "Park źródliska II" "" "Plastikowe butelki w Lasku przy osiedlu" ...
##  $ 7_Dzisiejsza_data__p             : chr  "16/06/2022" "16/06/2022" "12/06/2022" "10/06/2022" ...
##  $ 8_Aktualny_czas__pro             : chr  "10:54:39" "10:50:56" "16:00:03" "12:33:07" ...
##  $ 9_Czy_chcesz_nam_pow             : chr  "Tak" "Tak" "Nie" "Tak" ...
##  $ 10_Jak_oceniasz_powi             : chr  "od 6 do 50 m. kw - pokój lub kilka pokoi" "trudno powiedzieć" "" "do 5 metrów kwadratowych (m. kw) - mniejsze niż mały pokój" ...
##  $ 11_Jaki_jest_charakt             : chr  "drobne odpady rozproszone" "pojedyncze duże gabaryty (np. muszla klozetowa, lodówka)" "" "zwarty stos odpadów" ...
##  $ 12_Opisz_charakter_z             : chr  "" "" "" "" ...
##  $ 13_Czy_moesz_oceni_c             : chr  "powstałe w ostatnich miesiącach" "powstałe w ostatnich miesiącach" "" "powstałe w ostatnich dniach" ...
##  $ 14_W_jakim_miejscu_j             : chr  "teren zielony w mieście (zieleń miejska, park, skwer)" "teren zielony w mieście (zieleń miejska, park, skwer)" "" "teren zielony w mieście (zieleń miejska, park, skwer)" ...
##  $ 15_Opisz_miejsce_zna             : chr  "" "" "" "" ...
##  $ 16_Czy_znalezione_wy             : chr  "Dobrze widoczne" "Dobrze widoczne" "" "Dobrze widoczne" ...
##  $ 17_Jakie_rodzaje_odp             :List of 183
##   ..$ : chr  "folie" "plastik" "szkło " "butelki po alkoholu"
##   ..$ : chr "inne"
##   ..$ : chr 
##   ..$ : chr "plastik"
##   ..$ : chr  "przeterminowane środki ochrony roślin" "plastik" "tekstylia" "szkło " ...
##   ..$ : chr 
##   ..$ : chr 
##   ..$ : chr 
##   ..$ : chr  "plastik" "szkło " "zużyte opony" "meble" ...
##   ..$ : chr 
##   ..$ : chr 
##   ..$ : chr 
##   ..$ : chr 
##   ..$ : chr  "gruz poremontowy, pobudowlany" "zużyte opony" "folie" "szkło " ...
##   ..$ : chr 
##   ..$ : chr 
##   ..$ : chr 
##   ..$ : chr 
##   ..$ : chr 
##   ..$ : chr  "meble" "elektroodpady (zużyty sprzęt elektryczny i elektroniczny) " "plastik" "tekstylia" ...
##   ..$ : chr  "gruz poremontowy, pobudowlany" "meble" "zużyte opony" "folie" ...
##   ..$ : chr 
##   ..$ : chr 
##   ..$ : chr 
##   ..$ : chr 
##   ..$ : chr 
##   ..$ : chr 
##   ..$ : chr 
##   ..$ : chr 
##   ..$ : chr 
##   ..$ : chr 
##   ..$ : chr 
##   ..$ : chr 
##   ..$ : chr "części samochodowe"
##   ..$ : chr 
##   ..$ : chr 
##   ..$ : chr 
##   ..$ : chr 
##   ..$ : chr 
##   ..$ : chr 
##   ..$ : chr 
##   ..$ : chr  "gruz poremontowy, pobudowlany" "plastik" "szkło " "złom metalowy" ...
##   ..$ : chr  "gruz poremontowy, pobudowlany" "elektroodpady (zużyty sprzęt elektryczny i elektroniczny) " "folie" "butelki po alkoholu" ...
##   ..$ : chr 
##   ..$ : chr 
##   ..$ : chr 
##   ..$ : chr  "elektroodpady (zużyty sprzęt elektryczny i elektroniczny) " "inne"
##   ..$ : chr  "meble" "tekstylia" "butelki po alkoholu" "inne"
##   ..$ : chr 
##   ..$ : chr 
##   ..$ : chr 
##   ..$ : chr 
##   ..$ : chr  "zużyte opony" "tekstylia" "części samochodowe" "śmieci komunalne/ przydomowe – powstające w codziennym funkcjonowaniu gospodarstwa domowego" ...
##   ..$ : chr 
##   ..$ : chr 
##   ..$ : chr 
##   ..$ : chr 
##   ..$ : chr 
##   ..$ : chr 
##   ..$ : chr 
##   ..$ : chr  "plastik" "tekstylia" "folie" "szkło " ...
##   ..$ : chr 
##   ..$ : chr 
##   ..$ : chr 
##   ..$ : chr 
##   ..$ : chr  "zużyte opony" "części samochodowe" "folie" "plastik" ...
##   ..$ : chr  "gruz poremontowy, pobudowlany" "meble" "elektroodpady (zużyty sprzęt elektryczny i elektroniczny) " "zużyte opony" ...
##   ..$ : chr 
##   ..$ : chr 
##   ..$ : chr 
##   ..$ : chr 
##   ..$ : chr 
##   ..$ : chr 
##   ..$ : chr 
##   ..$ : chr 
##   ..$ : chr 
##   ..$ : chr 
##   ..$ : chr 
##   ..$ : chr 
##   ..$ : chr 
##   ..$ : chr 
##   ..$ : chr 
##   ..$ : chr 
##   ..$ : chr 
##   ..$ : chr 
##   ..$ : chr  "gruz poremontowy, pobudowlany" "zużyte opony" "części samochodowe" "folie" ...
##   ..$ : chr 
##   ..$ : chr 
##   ..$ : chr  "plastik" "puszki" "butelki po alkoholu" "folie"
##   ..$ : chr 
##   ..$ : chr 
##   ..$ : chr  "armatura łazienkowa (umywalka, sedes, wanna)" "elektroodpady (zużyty sprzęt elektryczny i elektroniczny) " "plastik" "folie" ...
##   ..$ : chr 
##   ..$ : chr 
##   ..$ : chr  "gruz poremontowy, pobudowlany" "armatura łazienkowa (umywalka, sedes, wanna)" "folie" "plastik" ...
##   ..$ : chr 
##   ..$ : chr 
##   ..$ : chr 
##   ..$ : chr 
##   .. [list output truncated]
##  $ 18_Opisz_rodzaj_znal             : chr  "Drobne śmieci" "Kosze na śmieci z parku wrzucone do stawu" "" "Butelki plastikowe po wodzie a także butelki po alkoholu. W Lasku przy osiedlu na ulicy Leszczyńskiej." ...
##  $ 19_Czy_po_raz_pierws             : chr  "Tak" "Nie" "Nie" "Nie" ...
##  $ 20_Wymyl_swoj_nazw_u             : chr  "Ula Urszula" "" "" "" ...
##  $ 21_Wpisz_swoj_nazw_u             : chr  "" "Gargl" "Katarzyna Wasilewska" "Ktk" ...
##  $ 22_Czy_chcesz_powied             : chr  "Nie, na dziś kończymy" "" "" "" ...
##  $ 5_Tutaj_zlokalizowan.latitude    : chr  "51.758141" "51.758051" "51.775315" "51.703427" ...
##  $ 5_Tutaj_zlokalizowan.longitude   : chr  "19.474411" "19.474483" "19.535988" "19.478217" ...
##  $ 5_Tutaj_zlokalizowan.accuracy    : chr  "5" "9" "6" "5" ...
##  $ 5_Tutaj_zlokalizowan.UTM_Northing: chr  "5735239" "5735229" "5737062" "5729149" ...
##  $ 5_Tutaj_zlokalizowan.UTM_Easting : chr  "394706" "394711" "398994" "394842" ...
##  $ 5_Tutaj_zlokalizowan.UTM_Zone    : chr  "34U" "34U" "34U" "34U" ...
##  $ date_user                        : Date, format: "2022-06-16" "2022-06-16" ...
##  $ volunteer_id                     : chr  "ula urszula" "gargl" "katarzyna wasilewska" "ktk" ...
##  $ latitude                         : num  51.8 51.8 51.8 51.7 51.8 ...
##  $ longitude                        : num  19.5 19.5 19.5 19.5 19.5 ...
\end{verbatim}

\begin{verbatim}
##                               ec5_uuid                              title
## 1 4302357d-bd62-461b-92cc-a9a65b9e7e92                05/05/2022 17:57:13
## 2 634f32ce-f43b-4409-b81c-5262750e0f44 16/04/2022 12:21:04 sebastian33444
## 3 7931f210-adf2-11ec-899c-29c766ce896f                27/03/2022 19:24:39
## 4 c4108f60-ac55-11ec-b78f-9f78d7afe582           25/03/2022 17:11:19 Juha
## 5 81229fc0-a95f-11ec-8b6c-857d0f1bfcdb                21/03/2022 22:49:31
## 6 2d9906cf-955e-4181-9e02-b2f0f16afe4b       10/03/2022 12:52:12 Pkrupnik
##                                           10_Jak_oceniasz_powi
## 1 większe niż 501 m. kw - plac Dąbrowskiego, boisko piłkarskie
## 2 większe niż 501 m. kw - plac Dąbrowskiego, boisko piłkarskie
## 3 większe niż 501 m. kw - plac Dąbrowskiego, boisko piłkarskie
## 4 większe niż 501 m. kw - plac Dąbrowskiego, boisko piłkarskie
## 5 większe niż 501 m. kw - plac Dąbrowskiego, boisko piłkarskie
## 6 większe niż 501 m. kw - plac Dąbrowskiego, boisko piłkarskie
\end{verbatim}

\begin{verbatim}
## Simple feature collection with 172 features and 1 field
## Geometry type: POINT
## Dimension:     XY
## Bounding box:  xmin: 19.37081 ymin: 51.68758 xmax: 19.60378 ymax: 51.82345
## Geodetic CRS:  WGS 84
## First 10 features:
##            volunteer_id                  geometry
## 1           ula urszula POINT (19.47441 51.75814)
## 2                 gargl POINT (19.47448 51.75805)
## 3  katarzyna wasilewska POINT (19.53599 51.77531)
## 4                   ktk POINT (19.47822 51.70343)
## 5               bogusia  POINT (19.48184 51.7725)
## 6                   rmo POINT (19.50163 51.75604)
## 7                   rmo POINT (19.50874 51.75416)
## 8                 gargl POINT (19.46672 51.72514)
## 9        sebastian33444 POINT (19.58445 51.74919)
## 10                  rmo  POINT (19.4646 51.75279)
\end{verbatim}

\hypertarget{cechy-dzikich-wysypisk-charakter}{%
\subsection{Cechy dzikich wysypisk --
charakter}\label{cechy-dzikich-wysypisk-charakter}}

\includegraphics{data_exploration_files/figure-latex/cechy-charakter-1.pdf}

\hypertarget{inny-jaki-to-charakter-todo}{%
\subsubsection{Inny -- jaki to charakter?
TODO}\label{inny-jaki-to-charakter-todo}}

\hypertarget{cechy-dzikich-wysypisk-miejsce}{%
\subsection{Cechy dzikich wysypisk --
miejsce}\label{cechy-dzikich-wysypisk-miejsce}}

\includegraphics{data_exploration_files/figure-latex/cechy-miejsce-1.pdf}

\hypertarget{inne-jakie-to-miejsce-todo}{%
\subsubsection{Inne -- jakie to miejsce?
TODO}\label{inne-jakie-to-miejsce-todo}}

\hypertarget{charakter-vs.-powierzchnia}{%
\subsubsection{Charakter
vs.~powierzchnia}\label{charakter-vs.-powierzchnia}}

Jak\_oceniasz\_powi

Jaki\_jest\_charakt

Total

drobne odpadyrozproszone

inne

kilka stosów

pojedyncze dużegabaryty (np. muszlaklozetowa, lodówka)

zwarty stos odpadów

do 5 metrówkwadratowych (m. kw)- mniejsze niż małypokój

{4}

{2}

{3}

{2}

{5}

{16}

od 51 do 500 m. kw -mieszkanie lubboisko do koszykówki

{6}

{0}

{6}

{1}

{0}

{13}

od 6 do 50 m. kw -pokój lub kilkapokoi

{3}

{3}

{6}

{0}

{2}

{14}

trudno powiedzieć

{0}

{0}

{0}

{1}

{0}

{1}

większe niż 501 m.kw - placDąbrowskiego, boiskopiłkarskie

{0}

{3}

{1}

{1}

{1}

{6}

Total

{13}

{8}

{16}

{5}

{8}

{50}

χ2=28.422 · df=16 · Cramer's V=0.377 · Fisher's p=0.033

\hypertarget{cechy-dzikich-wysypisk-rodzaje-odpaduxf3w}{%
\subsection{Cechy dzikich wysypisk -- rodzaje
odpadów}\label{cechy-dzikich-wysypisk-rodzaje-odpaduxf3w}}

\includegraphics{data_exploration_files/figure-latex/cechy-rodzaje-1.pdf}

Pytanie wielokrotnego wyboru, odpowiedzi nie sumują się do 50 wpisów.

\hypertarget{cechy-dzikich-wysypisk-widocznoux15bux107}{%
\subsection{Cechy dzikich wysypisk --
widoczność}\label{cechy-dzikich-wysypisk-widocznoux15bux107}}

\includegraphics{data_exploration_files/figure-latex/cechy-widocznosc-1.pdf}

Nikt nie wskazał, że wysypisko jest ``niewidoczne (ukryte)''.

\hypertarget{cechy-dzikich-wysypisk-czas-powstania}{%
\subsection{Cechy dzikich wysypisk -- czas
powstania}\label{cechy-dzikich-wysypisk-czas-powstania}}

\includegraphics{data_exploration_files/figure-latex/cechy-czas-1.pdf}

\end{document}
